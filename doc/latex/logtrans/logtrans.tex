\documentclass[12pt]{article}

\newcommand{\colore}{\texttt{CoLoRe}}
\newcommand{\vx}{\textbf{x}}
\newcommand{\vk}{\textbf{k}}

\begin{document}
\title{Power spectrum of log-normal transformed fields in
  redshift-space}
\author{A Slosar}
\maketitle

The log-normal transformation that \colore\ does is to 
\begin{enumerate}
\item Take the linear field in real space and smooth it
  $P=P_L(k)e^{-k^2r^2}$
\item Sample from the linear field
\item Transform the liner field according to transformation
  \begin{equation}
    \rho_{\ln}(\vx) = \exp\left[b\delta(\vx)\right]
  \end{equation}
\item Move galaxies around based on linear velocities.
\end{enumerate}

The mean density of the transformed fied is given by
\begin{equation}
  \bar{\rho}_{\ln} = \int \frac{1}{\sqrt{2\pi\sigma^2}}e^{b\delta -
    \frac{\delta^2}{2\sigma^2}} d \delta= \exp{\frac{\sigma^2b^2}{2}},
\end{equation}
where $\sigma^2$ is the field variance (that is why smoothing
matters!) and the transformed overdensity is given by 
\begin{equation}
  \delta_{\ln} = \frac{\rho_{\ln}}{\bar{\rho}_{\ln}} -1 
\end{equation}



Based on this prescription, the transformed field in redshift-space
is, on large scales
\begin{equation}
  \delta_{\ln,s}(\vk) = \delta_{\ln}(\vk) + \beta \delta (\vk) \mu^2,
\end{equation}
where $\beta=f/b$ is the usual Kaiser RSD distortion parameter. Note
that the second $\delta$ multiplying Kaiser term is linear. This
might seem a bit counter-intuitive, but can be shown to be using
peak-background split argument. It also likely breaks down an smaller
scales, but this is not crucial for us.

We have that the auto power-spectrum is given by 
\begin{equation}
  P_s(\vk) = P_{\ln\ln}(\vk) + 2P_{\ln L}\beta (\vk)\mu^2 + \beta^2 P_{LL}(\vk)\mu^4
\end{equation}

Missing terms can be calculated via Fourier transform of the
correlation functions.
\begin{eqnarray}
  \xi_{\ln L} (r) &=& \int d \delta_1 \int  d \delta_2 G(\delta_1, \delta_2|
  \sigma^2, \xi(r)) \delta_{\ln}(\delta_1) \delta_2\\
  \xi_{\ln \ln} (r) &=& \int d \delta_1 \int  d \delta_2 G(\delta_1, \delta_2|
  \sigma^2, \xi(r)) \delta_{\ln}(\delta_1) \delta_{\ln}(\delta_2),
\end{eqnarray}
where Gaussian $G$ describes correlations between two points in the
Gaussian field.
Doing the math, one gets, see the maple script \texttt{lntrans.mw}:
\begin{eqnarray}
  \xi_{\ln L} (r) &=& b\xi(r)\\
  \xi_{\ln \ln} (r) &=& e^{b^2 \xi(r)}-1
\end{eqnarray}
So, to make these predictions, one needs to logFFT to $\xi$, apply the
above transforms and logFFT back. Interesting the $L \ln$ terms
behaves exactly as if no transformation was done!

Note however, knowing the three power spectra allows one to make
predictions in Fourier space. To do the same in real space would
require to go multipoles and transform $\ell=0,2,4$ separately.

This has now been implemented into \colore. 



\end{document}
